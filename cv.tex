%%%%%%%%%%%%%%%%%
% This CV created using altacv.cls
% (v1.6.4, 13 Nov 2021) written by LianTze Lim (liantze@gmail.com). Now compiles with pdfLaTeX, XeLaTeX and LuaLaTeX.
%
%% It may be distributed and/or modified under the
%% conditions of the LaTeX Project Public License, either version 1.3
%% of this license or (at your option) any later version.
%% The latest version of this license is in
%%    http://www.latex-project.org/lppl.txt
%% and version 1.3 or later is part of all distributions of LaTeX
%% version 2003/12/01 or later.
%%%%%%%%%%%%%%%%

%% Use the "normalphoto" option instead of "ragged2e,withhyper" if you want a normal photo instead of cropped to a circle
\documentclass[10pt,a4paper,ragged2e,withhyper]{altacv}
%% AltaCV uses the fontawesome5 and packages.
%% See http://texdoc.net/pkg/fontawesome5 for full list of symbols.

% Change the page layout if you need to
\geometry{left=1.25cm,right=1.25cm,top=1.5cm,bottom=1.5cm,columnsep=1.2cm}

% The paracol package lets you typeset columns of text in parallel
\usepackage{paracol}

% Change the font if you want to, depending on whether
% you're using pdflatex or xelatex/lualatex
\ifxetexorluatex
  % If using xelatex or lualatex:
  \setmainfont{Roboto Slab}
  \setsansfont{Lato}
  \renewcommand{\familydefault}{\sfdefault}
\else
  % If using pdflatex:
  \usepackage[rm]{roboto}
  \usepackage[defaultsans]{lato}
  % \usepackage{sourcesanspro}
  \renewcommand{\familydefault}{\sfdefault}
\fi

% Change the colours if you want to
% \definecolor{SlateGrey}{HTML}{2E2E2E}
% \definecolor{LightGrey}{HTML}{666666}
% \definecolor{DarkPastelRed}{HTML}{450808}
% \definecolor{PastelRed}{HTML}{8F0D0D}
% \definecolor{GoldenEarth}{HTML}{E7D192}
\colorlet{name}{black}
\colorlet{tagline}{black}
\colorlet{heading}{black}
\colorlet{headingrule}{black}
\colorlet{subheading}{black}
\colorlet{accent}{black}
\colorlet{emphasis}{black}
\colorlet{body}{black}

% Change some fonts, if necessary
\renewcommand{\namefont}{\Huge\rmfamily\bfseries}
\renewcommand{\personalinfofont}{\footnotesize}
\renewcommand{\cvsectionfont}{\LARGE\rmfamily\bfseries}
\renewcommand{\cvsubsectionfont}{\large\bfseries}


% Change the bullets for itemize and rating marker
% for \cvskill if you want to
\renewcommand{\itemmarker}{{\small\textbullet}}
\renewcommand{\ratingmarker}{\faCircle}

\begin{document}
\name{Ekaterina Ivanova}
\tagline{CI/CD Devops Engineer}

%% You can add multiple photos on the left or right
% \photoR{2.8cm}{Globe_High}
% \photoL{2.5cm}{Yacht_High,Suitcase_High}

\personalinfo{
  \email{katenika158@gmail.com}
%   \phone{+79117500058}
  \linkedin{kate-iv}
  \github{buulka}
  %% You can add your own arbitrary detail with
  %% \printinfo{symbol}{detail}[optional hyperlink prefix]
  % \printinfo{\faPaw}{Hey ho!}[https://example.com/]
  %% Or you can declare your own field with
  %% \NewInfoFiled{fieldname}{symbol}[optional hyperlink prefix] and use it:
  % \NewInfoField{gitlab}{\faGitlab}[https://gitlab.com/]
  % \gitlab{your_id}
  %%
  %% For services and platforms like Mastodon where there isn't a
  %% straightforward relation between the user ID/nickname and the hyperlink,
  %% you can use \printinfo directly e.g.
  % \printinfo{\faMastodon}{@username@instace}[https://instance.url/@username]
}

\makecvheader
%% Depending on your tastes, you may want to make fonts of itemize environments slightly smaller
% \AtBeginEnvironment{itemize}{\small}

%% Set the left/right column width ratio to 6:4.
\columnratio{0.6}

% Start a 2-column paracol. Both the left and right columns will automatically
% break across pages if things get too long.
\begin{paracol}{2}
\cvsection{Experience}

\cvevent{CI/CD Engineer}{LoyaltyPlant}{Dec 2022 -- Ongoing}{}

\begin{itemize}
\item Designed CI/CD architecture in Jenkins and GitLab CI
\item Configured automated build and release pipelines for iOS and Android applications
\item Scaled build infrastructure (on-premise/cloud agents, optimization, maintenance)
\item Scripting experience in Python, Bash, and Ruby for automation and tooling
\item Python development (tools, automation, integrations)

\end{itemize}
Jenkins, Gitlab CI, Python, Bash, Ruby

\divider

\cvevent{Python Developer + DevOps}{Iqutor}{Apr 2022 -- Jan 2023}{}

\begin{itemize}
\item Implemented CI/CD pipelines and Docker-based deployment
\item Integrated and maintained Celery with RabbitMQ for task queues
\item Optimized system architecture by introducing asynchronous workflows
\item Managed containerized workloads using Kubernetes
\item Backend development in Python (features, bug fixes, optimizations)
\item Increased test coverage by 20% (unit/integration tests, pytest)
\item Refactored legacy codebase, reducing technical debt and doubling feature delivery speed

\end{itemize}
Django, PostgreSQL, Docker, RabbitMQ, Celery, Gitlab CI

\divider

\cvevent{Intern Backend Developer}{RobotBull Techonologies}{Oct 2021 - Dec 2021}{}

\begin{itemize}
    \item Internship in backend development using TypeScript and Nest.js, using Docker, Kubernetes, PostgreSQL and Minio (S3-compliant storage)
\end{itemize}

%% Switch to the right column. This will now automatically move to the second
%% page if the content is too long.
\switchcolumn

\cvsection{Education}

\cvevent{Bachelor's degree,\\ Software Engineering}{ITMO University}{}{}

\cvsection{Skills}

\cvevent{Programming Languages}{}{}{}

\cvtag{Python}
\cvtag{TypeScript}
\cvtag{SQL}

\divider\smallskip

\cvevent{Frameworks}{}{}{}

\cvtag{Django}
\cvtag{Flask}
\cvtag{SQL Alchemy}
\cvtag{Vue.js}


\divider\smallskip

\cvevent{Technologies}{}{}{}

\cvtag{PostgreSQL}
\cvtag{Celery}
\cvtag{RabbitMQ}
\cvtag{Minio}
\cvtag{Redis}

\divider\smallskip

\cvevent{Tools}{}{}{}

\cvtag{Docker}
\cvtag{Docker Compose}
\cvtag{Linux}
\cvtag{Git}
\cvtag{GitLab CI}

\cvsection{Languages}

\begin{itemize}
\item English - advanced
\item German - beginner
\item Russian - native
\end{itemize}

\end{paracol}

\end{document}
