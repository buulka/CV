%%%%%%%%%%%%%%%%%
% This CV created using altacv.cls
% (v1.6.4, 13 Nov 2021) written by LianTze Lim (liantze@gmail.com). Now compiles with pdfLaTeX, XeLaTeX and LuaLaTeX.
%
%% It may be distributed and/or modified under the
%% conditions of the LaTeX Project Public License, either version 1.3
%% of this license or (at your option) any later version.
%% The latest version of this license is in
%%    http://www.latex-project.org/lppl.txt
%% and version 1.3 or later is part of all distributions of LaTeX
%% version 2003/12/01 or later.
%%%%%%%%%%%%%%%%

%% Use the "normalphoto" option instead of "ragged2e,withhyper" if you want a normal photo instead of cropped to a circle
\documentclass[10pt,a4paper,ragged2e,withhyper]{altacv}
%% AltaCV uses the fontawesome5 and packages.
%% See http://texdoc.net/pkg/fontawesome5 for full list of symbols.

% Change the page layout if you need to
\geometry{left=1.25cm,right=1.25cm,top=1.5cm,bottom=1.5cm,columnsep=1.2cm}

% The paracol package lets you typeset columns of text in parallel
\usepackage{paracol}
\usepackage[T2A]{fontenc}
\usepackage[utf8]{inputenc}		
\usepackage[english,russian]{babel}

% Change the font if you want to, depending on whether
% you're using pdflatex or xelatex/lualatex
\ifxetexorluatex
  % If using xelatex or lualatex:
  \setmainfont{Roboto Slab}
  \setsansfont{Lato}
  \renewcommand{\familydefault}{\sfdefault}
\else
  % If using pdflatex:
  \usepackage[rm]{roboto}
  \usepackage[defaultsans]{lato}
  % \usepackage{sourcesanspro}
  \renewcommand{\familydefault}{\sfdefault}
\fi

% Change the colours if you want to
% \definecolor{SlateGrey}{HTML}{2E2E2E}
% \definecolor{LightGrey}{HTML}{666666}
% \definecolor{DarkPastelRed}{HTML}{450808}
% \definecolor{PastelRed}{HTML}{8F0D0D}
% \definecolor{GoldenEarth}{HTML}{E7D192}
\colorlet{name}{black}
\colorlet{tagline}{black}
\colorlet{heading}{black}
\colorlet{headingrule}{black}
\colorlet{subheading}{black}
\colorlet{accent}{black}
\colorlet{emphasis}{black}
\colorlet{body}{black}

% Change some fonts, if necessary
\renewcommand{\namefont}{\Huge\rmfamily\bfseries}
\renewcommand{\personalinfofont}{\footnotesize}
\renewcommand{\cvsectionfont}{\LARGE\rmfamily\bfseries}
\renewcommand{\cvsubsectionfont}{\large\bfseries}


% Change the bullets for itemize and rating marker
% for \cvskill if you want to
\renewcommand{\itemmarker}{{\small\textbullet}}
\renewcommand{\ratingmarker}{\faCircle}

\begin{document}
\name{Екатерина Иванова}
\tagline{CI/CD Devops Engineer}

%% You can add multiple photos on the left or right
% \photoR{2.8cm}{Globe_High}
% \photoL{2.5cm}{Yacht_High,Suitcase_High}

\personalinfo{
  \email{katenika158@gmail.com}
%   \phone{+79117500058}
  \linkedin{kate-iv}
  \github{buulka}
  %% You can add your own arbitrary detail with
  %% \printinfo{symbol}{detail}[optional hyperlink prefix]
  % \printinfo{\faPaw}{Hey ho!}[https://example.com/]
  %% Or you can declare your own field with
  %% \NewInfoFiled{fieldname}{symbol}[optional hyperlink prefix] and use it:
  % \NewInfoField{gitlab}{\faGitlab}[https://gitlab.com/]
  % \gitlab{your_id}
  %%
  %% For services and platforms like Mastodon where there isn't a
  %% straightforward relation between the user ID/nickname and the hyperlink,
  %% you can use \printinfo directly e.g.
  % \printinfo{\faMastodon}{@username@instace}[https://instance.url/@username]
}

\makecvheader
%% Depending on your tastes, you may want to make fonts of itemize environments slightly smaller
% \AtBeginEnvironment{itemize}{\small}

%% Set the left/right column width ratio to 6:4.
\columnratio{0.6}

% Start a 2-column paracol. Both the left and right columns will automatically
% break across pages if things get too long.
\begin{paracol}{2}
\cvsection{Опыт}

\cvevent{CI/CD Engineer}{LoyaltyPlant}{Дек 2022 -- Настоящее время}{}

\begin{itemize}
  \itemРазработка и внедрение архитектуры CI/CD-пайплайнов в Jenkins и GitLab CI;
  \itemАвтоматизация сборки и публикации мобильных приложений (iOS/Android);
  \itemОптимизация и масштабирование инфраструктуры сборки (агенты, кеширование, параллельные задачи);
  \itemСоздание скриптов для автоматизации (Python, Bash, Ruby);
  \itemРазработка инструментов и утилит на Python.
\end{itemize}
Jenkins, Gitlab CI, Python, Bash, Ruby

\divider

\cvevent{Python Developer + DevOps}{Iqutor}{Апр 2022 -- Янв 2023}{}

\begin{itemize}
  \itemНастройка CI/CD-процессов и деплой в Docker-окружении;
  \itemИнтеграция и настройка брокеров сообщений: Kafka, RabbitMQ;
  \itemРефакторинг монолитной архитектуры с переходом на асинхронную модель взаимодействия;
  \itemРазвертывание и оркестрация микросервисов в Kubernetes;
  \itemРазработка backend-компонентов на Python;
  \itemПовышение test coverage на 20% за счет юнит- и интеграционных тестов;
  \itemМодернизация legacy-кода, что сократило время разработки новых функций в 2 раза.
\end{itemize}
Django, PostgreSQL, Docker, RabbitMQ, Celery, Gitlab CI

\divider

\cvevent{Intern Backend Developer}{RobotBull Techonologies}{Окт 2021 - Дек 2021}{}

\begin{itemize}
  \itemПрактика разработки серверной части на TypeScript с использованием Nest.js;
  \itemРабота с контейнеризацией Docker и оркестрацией Kubernetes:
  \itemПроектирование и оптимизация запросов в PostgreSQL;
  \itemРеализация хранения файлов с использованием Minio (S3-compliant storage);
  \itemУчастие в разработке API и бизнес-логики сервиса.
\end{itemize}

%% Switch to the right column. This will now automatically move to the second
%% page if the content is too long.
\switchcolumn

\cvsection{Образование}

\cvevent{Бакалавриат,\\ Прикладная информатика}{Университет ITMO}{}{}

\cvsection{Навыки}

\cvevent{Языки программирования}{}{}{}

\cvtag{Python}
\cvtag{Bash}
\cvtag{Ruby}
\cvtag{SQL}

\divider\smallskip

\cvevent{Frameworks}{}{}{}

\cvtag{Django}
\cvtag{Flask}
\cvtag{SQL Alchemy}
\cvtag{Vue.js}


\divider\smallskip

\cvevent{Технологии}{}{}{}

\cvtag{PostgreSQL}
\cvtag{Celery}
\cvtag{RabbitMQ}
\cvtag{Minio}
\cvtag{Redis}

\divider\smallskip

\cvevent{Инструменты}{}{}{}

\cvtag{Docker}
\cvtag{Docker Compose}
\cvtag{Jenkins}
\cvtag{Linux}
\cvtag{Git}
\cvtag{GitLab CI}

\cvsection{Языки}

\begin{itemize}
\item Англиский - продвинутый
\item Немецкий - начинающий
\item Русский - носитель
\end{itemize}

\end{paracol}

\end{document}
